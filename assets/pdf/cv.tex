\documentclass[10pt,a4paper,calibri]{moderncv}
\usepackage[left=0.75in,right=0.75in,top=0.75in,bottom=0.75in]{geometry}
\usepackage{url}
\usepackage{tikz}
\usepackage{ragged2e}
\usepackage{xstring}

\definecolor{blueviolet}{rgb}{0.4, 0.4, 1}
\definecolor{lightgray}{gray}{0.7}

\moderncvtheme[black]{banking}

% ######### Variable ############
\newcommand{\beforesection}{\vspace{-0.5em}}
\newcommand{\mediumspace}{\vspace{0.5em}}

% Education Details
\newcommand{\education}[2]{
  \noindent
  \begin{minipage}[c]{0.20\textwidth}
    \hfill
    #1
  \end{minipage}%
  \hspace{0.01\textwidth}
  \begin{minipage}[c]{0.02\textwidth}
    \textcolor{lightgray}{\rule{1pt}{1.6cm}}
  \end{minipage}%
  \begin{minipage}[c]{0.80\textwidth}
    \raggedright{
    #2
    }
  \end{minipage}%
}

% Work Experience Details
\newcommand{\workexperience}[2]{
  \noindent
  \begin{minipage}[c]{0.20\textwidth}
    \begin{flushright}
      #1      
    \end{flushright}
  \end{minipage}%
  \hspace{0.01\textwidth}
  \begin{minipage}[c]{0.02\textwidth}
    \textcolor{lightgray}{\rule{1pt}{0.7cm}}
  \end{minipage}%
  \begin{minipage}[c]{0.80\textwidth}
    \raggedright{
    #2
    }
  \end{minipage}%
}

% Work Experience Details
\newcommand{\publications}[2]{
  \noindent
  \begin{minipage}[c]{0.15\textwidth}
    \centering
      \textbf{#1}
  \end{minipage}%
  \hspace{0.01\textwidth}
  \begin{minipage}[c]{0.02\textwidth}
    \textcolor{lightgray}{\rule{1pt}{0.75cm}}
  \end{minipage}%
  \begin{minipage}[c]{0.80\textwidth}
    \raggedright{
    #2
    }
  \end{minipage}%
}

% Teaching Experience Details
\newcommand{\teachingexperience}[2]{
\noindent
  \begin{minipage}[c]{0.12\textwidth}
    \begin{center}
        #1
    \end{center}
  \end{minipage}%
  \hspace{0.01\textwidth}
  \begin{minipage}[c]{0.02\textwidth}
    \textcolor{lightgray}{\rule{1pt}{0.7cm}}
  \end{minipage}%
  \begin{minipage}[c]{0.86\textwidth}
    \raggedright{
      \vspace{0.001cm}
      #2
      \vspace{0.001cm}
    }  
  \end{minipage}%
}

% Activities Details
\newcommand{\activities}[2]{
\noindent
  \begin{minipage}[c]{0.10\textwidth}
    \hfill
    #1
  \end{minipage}%
  \hspace{0.01\textwidth}
  \begin{minipage}[c]{0.02\textwidth}
    \textcolor{lightgray}{\rule{1pt}{0.4cm}}
  \end{minipage}%
  \begin{minipage}[c]{0.86\textwidth}
    \justify{
      \raggedright{
        #2
      }  
    }
  \end{minipage}%
}

% Certificate Details
\newcommand{\certificate}[2]{
\noindent
  \begin{minipage}[c]{0.10\textwidth}
    \hfill
    #1
  \end{minipage}%
  \hspace{0.01\textwidth}
  \begin{minipage}[c]{0.02\textwidth}
    \textcolor{lightgray}{\rule{1pt}{0.4cm}}
  \end{minipage}%
  \begin{minipage}[c]{0.86\textwidth}
    \raggedright{
      #2
    }  
  \end{minipage}%
}

\name{Shubham}{Kaushik}
\extrainfo{PhD Student \textbf{@} Brandeis University}

\begin{document}

\makecvtitle\

\vspace{-50pt}
\section{Contact Information}

Contact.: \href{tel:+17745190913}{+1 (774) 519-0913}\\
Email: \href{mailto:kaushiks@brandeis.edu}{\textcolor{blueviolet}{kaushiks@brandeis.edu}}\;; 
\href{mailto:shubhamk00020@gmail.com}{\textcolor{blueviolet}{shubhamk00020@gmail.com}}\\
Website: \href{https://www.shubhamkaushik.com}{\textcolor{blueviolet}{shubhamkaushik.com}}\;; 
\href{https://www.linkedin.com/in/shubham-sudo}{\textcolor{blueviolet}{Linkedin}}\;; 
\href{https://www.github.com/shubham-sudo}{\textcolor{blueviolet}{Github}}
% Address: 569 Cambridge Street, Apt. \#2, Allston, MA 02134
\beforesection

\section{Research Interests}
\cvline{}{Databases, Data systems, Storage systems, Distributed systems, Data streaming, Cyber security}

\beforesection

\section{Work Experience}
\workexperience{Jan 2024 \-- Present}{
  \textbf{PhD Researcher}\\
  \href{https://www.brandeis.edu/}{\textcolor{blueviolet}{Brandeis University}}, MA, United States\\
}

\mediumspace

\workexperience{Mar 2022 \-- Aug 2022}{
  \textbf{Software Engineer}, \textit{Server Programming Team}\\
    \href{https://www.kwalee.com/}{\textcolor{blueviolet}{Kwalee}}, India\\
}

\mediumspace

\workexperience{Jun 2021 \-- Mar 2022}{
  \textbf{Engineer \-- Information Security}, \textit{Cyber Fusion, Information Security}\\
    \href{https://www.fisglobal.com/en}{\textcolor{blueviolet}{FIS Global}}, India\\
}

\mediumspace

\workexperience{Oct 2019 \-- Jun 2021}{
  \textbf{Project Engineer}, \textit{Python Cloud Computing, Wipro Digital}\\
    \href{https://www.wipro.com/}{\textcolor{blueviolet}{Wipro Limited}}, India\\
}

\mediumspace

\workexperience{Jul 2018 \-- Oct 2019}{
  \textbf{Project Engineer}, \textit{Big Data, Cyber Defense}\\
    \href{https://www.wipro.com/}{\textcolor{blueviolet}{Wipro Limited}}, India\\
}

\mediumspace

\workexperience{Mar 2017 \-- Apr 2017}{
  \textbf{Full Stack Developer Intern}, \textit{Backend Team}\\
    SoPo Internet Private Limited, India\\
}

\beforesection

\section{Teaching Experience}
\teachingexperience{\hfill Spring 2024}{
    \textbf{Teaching Assistant}, \textit{Database Management Systems (COSI 127B)}\\
    \textit{Michtom School of Computer Science}, Brandeis University, MA, United States\\
}

\mediumspace

\teachingexperience{\hfill Fall 2023\\ \hfill Spring 2023}{
    \textbf{Teaching Assistant}, \textit{Data Mechanics (DS 310)}\\
    \textit{Center for Computing \& Data Sciences}, Boston University, MA, United States\\
}

\mediumspace

\teachingexperience{\hfill Fall 2022}{
  \textbf{Teaching Assistant}, \textit{Computer Networks (CS 455)}\\
  \textit{Department of Computer Science}, Boston University, MA, United States\\
}

\beforesection

\section{Education}
\noindent
\begin{minipage}[c]{0.20\textwidth}
  \hfill
  Jan 2024 \-- Present
\end{minipage}%
\hspace{0.01\textwidth}
\begin{minipage}[c]{0.02\textwidth}
  \textcolor{lightgray}{\rule{1pt}{1.2cm}}
\end{minipage}%
\begin{minipage}[c]{0.80\textwidth}
  \raggedright{
    \textbf{Doctor of Philosophy (PhD)}\\
    \href{https://www.brandeis.edu/}{\textcolor{blueviolet}{Brandeis University}}, MA, United States\\
    Major: \textbf{Computer Science}\\
  }
\end{minipage}%

\mediumspace

\education{Sep 2022 \-- Dec 2023}
{
  \textbf{Masters of Science (M.S.)}\\
  \href{https://www.bu.edu/}{\textcolor{blueviolet}{Boston University}}, MA, United States\\
  Major: \textbf{Computer Science} with specialization in \textit{“Data-Centric Computing"}\\
  GPA: \textit{3.88/4.0}
}

\mediumspace

\education{Jul 2014 \-- Jun 2018}
{
  \textbf{Bachelor of Technology (B.Tech.)}\\
  \href{https://mdu.ac.in/}{\textcolor{blueviolet}{Maharshi Dayanand University}}, Haryana, India\\
  Major: \textbf{Computer Science} \& \textbf{Engineering}\\
  Thesis: \textit{“Fault Modelling of an Object-Oriented System using Colored Petri Nets”}\\
}

\beforesection

\section{Publication}
\publications{DBTest 2024}{\textbf{Shubham Kaushik}, Subhadeep Sarkar \href{https://doi.org/10.1145/3662165.3662766}{\textit{\textcolor{blueviolet}{Anatomy 
of the LSM Memory Buffer: Insights \& Implications}}}, International Workshop on Testing Database Systems}

\mediumspace

\publications{JCSE 2019}{\textbf{Shubham Kaushik}, Ratneshwer.\ \href{https://doi.org/10.26438/ijcse/v7i5.18281845}{\textit{\textcolor{blueviolet}{Fault 
Modelling of an Object-Oriented System using CPN}}}, International Journal of Computer Sciences and Engineering}

\mediumspace
\beforesection

\section{Posters}
\publications{NEDB Day 2024}{\textbf{Shubham Kaushik}, Manos Athanassoulis, Subhadeep Sarkar \href{https://bu-disc.github.io/nedbday/2024/download/posters/RangeReduce_A_Range_Query_Driven_Compaction_for_LSM-Trees.pdf}{\textit{\textcolor{blueviolet}{RangeReduce: A Range Query Driven Compaction for LSM-Trees}}}, North East Database Day}

\beforesection

\section{Bachelor's Thesis}
\textbf{Shubham Kaushik}.\ \href{https://doi.org/10.26438/ijcse/v7i5.18281845}
{\textit{\textcolor{blueviolet}{Fault Modelling of an Object-Oriented System using Colored Petri Nets},}} 2018.\\
% Department of Computer Science \& Engineering, Maharshi Dayanand University, India.\\
Advisor: \href{https://www.jnu.ac.in/content/ratnesh}{\textcolor{blueviolet}{Dr.\ Ratneshwer}}, 
School of Computer and Systems Sciences, Jawaharlal Nehru University.

% \mediumspace

% My thesis analyzes the formal representation of object-oriented properties using colored petri nets to enhance the 
% understanding of software fault behavior, offering insights for the software development, testing, and 
% maintenance phases. I modeled the various faults in object-oriented systems, emphasizing on faults related to object 
% interactions, such as inheritance and polymorphism states.

\beforesection

\section{Technical Skills}
\begin{itemize}\setlength\itemsep{0.2em}
  \item \textbf{Programming Languages}: C, C++, Python, Java, SQL, Rust (\textit{learning})
  \item \textbf{Markup Languages}: HTML, CSS, JSON, YAML, \LaTeX, Markdown
  \item \textbf{Databases}: RocksDB, MySQL, MongoDB, Redis, SQLite, ORM
  \item \textbf{Tools \& Frameworks}: Django, Flask, Microservices, Asyncio, Kafka, Git, ETL, Flink, AWS
\end{itemize}

\beforesection

\section{Projects}

% \textit{\fontsize{12}{0}\selectfont \underline{Academic Research}}
\begin{itemize}\setlength\itemsep{0.3em}
  \item \textbf{Designing Range Query-Aware Log-Structured Merge (LSM) Trees} \textit{(Ongoing)}: Designing a new 
  family of data reorganization strategies and data layouts, driven by range queries.
  % LSM trees are at 
  % the heart of several NoSQL data stores due to their ingestion-optimized design. However, this superior ingestion 
  % performance comes at the cost of poor range query performance and increased write amplification. In this project, we 
  % introduce a new family of data reorganization strategies and data layouts, driven by range queries. These strategies allow us to (i) 
  % reduce the overall data movement during workload execution and (ii) reduce the I/O cost for future range queries. I am
  % currently in the process of integrating our solutions on RocksDB, a widely used commercial LSM-based data store.
  \href{https://drive.google.com/file/d/19AeEG3e_Tao52vNcxTtTuzZMYfD2s9Tx/view?usp=sharing}{[\textcolor{blueviolet}{ppt}]}
  \href{https://drive.google.com/file/d/1g9B3lbKy7yektDqqtbGGdYSRTl7-SEvC/view?usp=sharing}{[\textcolor{blueviolet}{report}]}
  \href{https://github.com/shubham-sudo/LSMQueryDrivenCompaction}{[\textcolor{blueviolet}{readme}]}


  \item \textbf{Enabling Efficient Range Deletes in LSM-Trees} \textit{(Ongoing)}: Introducing a lightweight and updatable range delete filter to avoid 
  superfluous accesses to storage using a small amount of metadata
  % LSM-based data stores perform 
  % data deletion logically, without physically deleting the target data objects. This leads to significant performance 
  % bottlenecks when deleting ranges of data, as the logically deleted data continues to `live' in the database, 
  % increasing the overall cost of operations. In this project, we introduce a light weight and updatable range delete filter to avoid 
  % superfluous accesses to slow storage in exchange for a small amount of metadata in memory. The proposed solution 
  % substantially reduces the execution cost for workloads with range deletes.
  \href{https://drive.google.com/file/d/1MNnnk1DBNMEPNbv99z1D3LM3pILTUgj1/view?usp=sharing}{[\textcolor{blueviolet}{report}]}
  \href{https://github.com/shubham-sudo/LSMRangeDeletes}{[\textcolor{blueviolet}{readme}]}


  \item \textbf{Multi Layered Detection Model (MLED) for Error Detection} \textit{(Ongoing)}: Designing an architecture 
  to significantly reduce the Undetected Error Probability in petabyte-scale file transfers.
  % MLED is a layered 
  % architecture designed to significantly reduce the Undetected Error Probability in file transfers. This is particularly 
  % important for petabyte-scale file transfers, which are often used for data collected from scientific instruments. The 
  % architecture is parameterized by a number of layers ($n$), and a policy ($P_i$) for each layer that describes its 
  % operation. MLED provides flexibility with variable layer counts in different network sections and allows distinct 
  % policies at various layers. Its recursive delivery system ensures a high probability of error-free file transfer to 
  % the destination.
  \href{https://docs.google.com/presentation/d/1W_REQ6JN4Z17PV5PPEUSalE829oVi6tP/edit?usp=drive_link&ouid=111208771990960831325&rtpof=true&sd=true}{[\textcolor{blueviolet}{ppt}]}
  \href{https://github.com/shubham-sudo/mled}{[\textcolor{blueviolet}{readme}]}


  \item \textbf{Heterogeneity-Aware Operator Placement for Stream Processing Systems at the Edge}: Proposing
  an approach to dynamically place operators based on the selectivity and heterogeneity of the data.
  % Streaming systems are 
  % widely used for real-time data processing. However, all operators within a cluster runs with a static configuration, which 
  % is suboptimal for dynamic workloads. In this project, we proposed an approach to dynamically place operators based on 
  % the selectivity and heterogeneity of the data. Toward this, I modified Apache Flink's scheduler to dynamically 
  % switch tasks at the edge devices (\textit{Raspberry Pi}) and servers. This reduced the network traffic and improved 
  % system efficiency and resource utilization.
  \href{https://docs.google.com/presentation/d/1aD28E6t7_gmsuiYY4dATWUFKkzRcipnO/edit?usp=sharing&ouid=111208771990960831325&rtpof=true&sd=true}{[\textcolor{blueviolet}{ppt}]}
  \href{https://github.com/shubham-sudo/HeterogeneityAwareOperatorPlacementforStreamProcessingSystemsAtEdge-RaspberryPi}{[\textcolor{blueviolet}{readme}]}


  \item \textbf{Finding Vulnerabilities in VS Code Extensions}: Devised a simulation framework using Pyautogui 
  to install and execute extensions, and detect security vulnerabilities by analyzing the open ports.
  % The use of third-party extensions can introduce 
  % potential security vulnerabilities, which can also render the base applications vulnerable. In this project, I identified 
  % security vulnerabilities in VS Code extensions and developed an automated tool for their detection. I devised a 
  % simulation framework using the Pyautogui library to install and execute extensions, and detect security vulnerabilities 
  % by analyzing the open ports associated with each extension. The analysis specifically targeted \textit{Path Traversal} and 
  % \textit{Zip Slip} attacks, and detected 5\% of the extensions examined as vulnerable. 
  % \href{https://docs.google.com/presentation/d/1XQZZJBEJJRjX5N6yPjL3KkFcKuOgMxaI/edit?usp=sharing&ouid=111208771990960831325&rtpof=true&sd=true}{[\textcolor{blueviolet}{ppt}]}
  \href{https://drive.google.com/file/d/1SUWsf2YVB4lKOOoEUgAEGc34aK5barhC/view?usp=sharing}{[\textcolor{blueviolet}{report}]}
  \href{https://github.com/prateekdceit06/VSCode-Extensions-Simulator}{[\textcolor{blueviolet}{readme}]}
\end{itemize}

% \mediumspace

% \textit{\fontsize{12}{0}\selectfont \underline{Kwalee}}
% \begin{itemize}\setlength\itemsep{0.3em}
%   \item \textbf{Leaderboard \& Loot-Box}: 
%   As a member of a cross-functional team, I played a crucial role in designing and developing 
%   two essential features for hyper-casual games: the \textit{leaderboard} and \textit{loot box}. These features were integrated into three new 
%   game releases, collaborating closely with the \textit{Tools \& Programming} team to ensure seamless functionality 
%   within the games. Moreover, I conducted thorough \textit{code reviews} to identify areas for improvement, including 
%   performance optimizations, adherence to coding standards, and bug fixing.
% \end{itemize}

% \mediumspace

% \textit{\fontsize{12}{0}\selectfont \underline{Industry Research}}
% \begin{itemize}\setlength\itemsep{0.3em}
%   \item \textbf{Asynchronous IP-Scanning}: In this project, I designed an \textit{asynchronous} IP scanning module to 
%   pipeline the scanning of a large number (more than 100K) of IPs from internet. This module was implemented in 
%   \textit{Python} while leveraging \textit{multithreading}, \textit{socket programming}, and \textit{asyncio} techniques 
%   to maximize the performance. The automation enhanced the security posture, emphasizing the importance of continuously 
%   monitoring IP addresses, and improved the productivity of the security team by up to 50\%.
  
%   % The automation of the IP Scanning process significantly enhanced the security posture, 
%   % emphasizing the importance of continuously monitoring and assessing IP addresses to identify potential vulnerabilities 
%   % and threats.
 
%   % \item \textbf{Playbooks \& Integrations (Palo Alto Cortex XSOAR)}: I developed integration 
%   % scripts and playbooks using the \textit{Palo Alto Networks Cortex XSOAR} platform to enhance security automation and 
%   % orchestration capabilities. These contributions, made to the \href{https://github.com/xsoar-contrib}{\textit{``xsoar-contrib"}} 
%   % repository, included integrations scripts for platforms such as 
%   % \href{https://github.com/demisto/content/pull/18139}{\textit{``\textcolor{blueviolet}{DB2}"}}, 
%   % \href{https://github.com/demisto/content/pull/17845}{\textit{``\textcolor{blueviolet}{GenericSQL}"}}, and 
%   % \href{https://github.com/demisto/content/pull/17455}{\textit{``\textcolor{blueviolet}{FireMon Security}"}}. These 
%   % integrations help security teams to connect such platforms with Cortex XSOAR, enabling log collection and 
%   % reinforcing overall security.
% \end{itemize}

% \mediumspace

% \begin{itemize}\setlength\itemsep{0.3em}
%   % \item \textbf{Data Masking}: I developed a RESTful microservice to securely mask personally identifiable information 
%   % in various file formats, including CSV, Excel, mainframe, and HTML.\ I also designed an efficient masking algorithm 
%   % that pseudonymizes the users' personal data stored in flat files. This service streamlined the process of masking 
%   % thousands of files, enhancing the system performance and reducing manual intervention by 70\%.
  
%   % \item \textbf{Data Inquiry}: I optimized the Data Inquiry service using the \textit{asyncio framework} in Python 
%   % which enabled \textit{asynchronous execution} and effective coordination of HTTP requests. It allowed the inquiry 
%   % service to process multiple requests concurrently and reduce the turnaround time. These enhancements improved the 
%   % system efficiency by up to 50\%, allowing faster responses and better utilization of computational resources.
  
%   \item \textbf{Anomaly Detection}: In this cyber-defense project, I developed a novel anomaly detection agent 
%   that is capable of monitoring network traffic from multiple systems using \textit{Scapy}. The agent continuously 
%   analyzed the data in real time, identifying and flagging any suspicious or unusual activities. The agent also 
%   identified malicious IPs using \textit{K-means clustering} and a repository of malicious 
%   IPs. It triggers alerts for harmful events, providing valuable information to security analysts for further 
%   investigation and threat mitigation.
  
%   % \item \textbf{Incident Management System}: I developed and implemented a comprehensive incident 
%   % management system to streamline the tracking and prioritization of compromised incidents. The system 
%   % allows efficient categorization of incidents based on their severity levels, allowing the \textit{security engineering} 
%   % team to prioritize and respond promptly to critical security threats. This implementation improved the security
%   % engineering team response time by 40\% for critical incidents.
% \end{itemize}

\beforesection

\section{Certifications}

\certificate{Jul 2023}{
  \textit{``The Ultimate Hands-On \textbf{Hadoop}: Tame your \textbf{Big Data!}''} \-- Udemy 
  \href{https://ude.my/UC-5946a326-519d-4821-8901-aa214407757c}{[\textcolor{blueviolet}{link}]}
}

\mediumspace

\certificate{Jul 2023}{
  \textit{``Beginning \textbf{C++} programming from Beginner to Beyond''} \-- Udemy
  \href{https://ude.my/UC-26cecb9c-69d3-4d6c-a8bc-6167d0b145d4}{[\textcolor{blueviolet}{link}]}
}

\mediumspace

\certificate{Oct 2018}{
  Statement of accomplishment for \textit{``\textbf{Python} Track''} \-- DataCamp
  \href{https://www.datacamp.com/statement-of-accomplishment/track/5c7d7a07e88eef902cc101bc72ebf219dc312b1c}
  {[\textcolor{blueviolet}{link}]}
}

\mediumspace

\certificate{May 2016}{
  \textit{``Core \textbf{Java}''} \-- Oracle's Workforce Development Program
  \href{https://drive.google.com/file/d/1fl1wQYMtamojp-9NRT6B2itN1w1XTzFG/view?usp=sharing}{[\textcolor{blueviolet}{link}]}
}

\beforesection

\section{Curricular Activities}

\activities{Sep 2023}{
  Judged and mentored at \href{https://hackmit.org/}{\textcolor{blueviolet}{\textit{HackMIT 2023}}}, aiding teams with 
  technical challenges.
}

\mediumspace

\activities{Nov 2022}{
  Mentored 4 teams, with an average of 20 participants at \href{https://bostonhacks.io/}
  {\textit{\textcolor{blueviolet}{BostonHacks}}}.
}

\mediumspace

\activities{Jan 2017}{
  Volunteered in the Program Event Management team at the \textit{National Youth Festival}.
}

\end{document}
