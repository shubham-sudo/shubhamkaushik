\begin{itemize}\setlength\itemsep{0.4em}
	\item \textbf{Range Query-Aware Log-Structured Merge (LSM) Trees} \textit{(Ongoing)}: Developing range query driven compaction strategies to optimize overall system performance of LSM-based data stores.
	      \href{https://github.com/shubham-sudo/LSMQueryDrivenCompaction}{[\textcolor{blueviolet}{readme}]}
	      % LSM trees are at the heart of several NoSQL data stores due to their ingestion-optimized design. However, this superior ingestion 
	      % performance comes at the cost of poor range query performance and increased write amplification. In this project, we 
	      % introduce a new family of data reorganization strategies and data layouts, driven by range queries. These strategies allow us to (i) 
	      % reduce the overall data movement during workload execution and (ii) reduce the I/O cost for future range queries. I am
	      % currently in the process of integrating our solutions on RocksDB, a widely used commercial LSM-based data store.
	      % \href{https://drive.google.com/file/d/19AeEG3e_Tao52vNcxTtTuzZMYfD2s9Tx/view?usp=sharing}{[\textcolor{blueviolet}{ppt}]}
	      % \href{https://drive.google.com/file/d/1g9B3lbKy7yektDqqtbGGdYSRTl7-SEvC/view?usp=sharing}{[\textcolor{blueviolet}{report}]}

	% \item \textbf{Enabling Efficient Range Deletes in LSM-Trees} \textit{(Ongoing)}: Introducing a lightweight and updatable range delete filter to avoid 
	      % superfluous accesses to storage using a small amount of metadata
	      % % LSM-based data stores perform 
	      % % data deletion logically, without physically deleting the target data objects. This leads to significant performance 
	      % % bottlenecks when deleting ranges of data, as the logically deleted data continues to `live' in the database, 
	      % % increasing the overall cost of operations. In this project, we introduce a light weight and updatable range delete filter to avoid 
	      % % superfluous accesses to slow storage in exchange for a small amount of metadata in memory. The proposed solution 
	      % % substantially reduces the execution cost for workloads with range deletes.
	      % \href{https://drive.google.com/file/d/1MNnnk1DBNMEPNbv99z1D3LM3pILTUgj1/view?usp=sharing}{[\textcolor{blueviolet}{report}]}
	      % \href{https://github.com/shubham-sudo/LSMRangeDeletes}{[\textcolor{blueviolet}{readme}]}

	\item \textbf{Multi Layered Detection Model (MLED) for Error Detection} \textit{(Ongoing)}: Creating a flexible system to reduce undetected errors in petabyte-scale file transfers through layered error-checking methods.
	      \href{https://github.com/shubham-sudo/mled}{[\textcolor{blueviolet}{readme}]}
				% MLED is a layered architecture designed to significantly reduce the Undetected Error Probability in file transfers. This is particularly 
	      % important for petabyte-scale file transfers, which are often used for data collected from scientific instruments. The 
	      % architecture is parameterized by a number of layers ($n$), and a policy ($P_i$) for each layer that describes its 
	      % operation. MLED provides flexibility with variable layer counts in different network sections and allows distinct 
	      % policies at various layers. Its recursive delivery system ensures a high probability of error-free file transfer to 
	      % the destination.
	      % \href{https://docs.google.com/presentation/d/1W_REQ6JN4Z17PV5PPEUSalE829oVi6tP/edit?usp=drive_link&ouid=111208771990960831325&rtpof=true&sd=true}{[\textcolor{blueviolet}{ppt}]}

	\item \textbf{Benchmarking LSM-Based Storage Engines}: Analyzed performance of LSM trees with different memory buffers across various types of workloads, offering guidelines for optimal buffer selection.
	      \href{https://shubhamkaushik.com/assets/pdf/LSMMemory.pdf}{[\textcolor{blueviolet}{publication}]}

	\item \textbf{Heterogeneity-Aware Operator Placement for Streaming Systems}: Proposed a dynamic method to place data processing operators based on data selectivity, improving efficiency and reducing network traffic.
	      \href{https://github.com/shubham-sudo/HeterogeneityAwareOperatorPlacementforStreamProcessingSystemsAtEdge-RaspberryPi}{[\textcolor{blueviolet}{readme}]}
				% Streaming systems are widely used for real-time data processing. However, all operators within a cluster runs with a static configuration, which 
	      % is suboptimal for dynamic workloads. In this project, we proposed an approach to dynamically place operators based on 
	      % the selectivity and heterogeneity of the data. Toward this, I modified Apache Flink's scheduler to dynamically 
	      % switch tasks at the edge devices (\textit{Raspberry Pi}) and servers. This reduced the network traffic and improved 
	      % system efficiency and resource utilization.
	      % \href{https://docs.google.com/presentation/d/1aD28E6t7_gmsuiYY4dATWUFKkzRcipnO/edit?usp=sharing&ouid=111208771990960831325&rtpof=true&sd=true}{[\textcolor{blueviolet}{ppt}]}

	\item \textbf{Finding Vulnerabilities in VS Code Extensions}: Created a simulation framework to automate the installation and execution of VS Code extensions, identifying security vulnerabilities by analyzing open ports.
	      \href{https://github.com/prateekdceit06/VSCode-Extensions-Simulator}{[\textcolor{blueviolet}{readme}]}
	      % The use of third-party extensions can introduce 
	      % potential security vulnerabilities, which can also render the base applications vulnerable. In this project, I identified 
	      % security vulnerabilities in VS Code extensions and developed an automated tool for their detection. I devised a 
	      % simulation framework using the Pyautogui library to install and execute extensions, and detect security vulnerabilities 
	      % by analyzing the open ports associated with each extension. The analysis specifically targeted \textit{Path Traversal} and 
	      % \textit{Zip Slip} attacks, and detected 5\% of the extensions examined as vulnerable. 
	      % \href{https://docs.google.com/presentation/d/1XQZZJBEJJRjX5N6yPjL3KkFcKuOgMxaI/edit?usp=sharing&ouid=111208771990960831325&rtpof=true&sd=true}{[\textcolor{blueviolet}{ppt}]}
	      % \href{https://drive.google.com/file/d/1SUWsf2YVB4lKOOoEUgAEGc34aK5barhC/view?usp=sharing}{[\textcolor{blueviolet}{report}]}

\end{itemize}

% \mediumspace

% \textit{\fontsize{12}{0}\selectfont \underline{Kwalee}}
% \begin{itemize}\setlength\itemsep{0.3em}
%   \item \textbf{Leaderboard \& Loot-Box}: 
%   As a member of a cross-functional team, I played a crucial role in designing and developing 
%   two essential features for hyper-casual games: the \textit{leaderboard} and \textit{loot box}. These features were integrated into three new 
%   game releases, collaborating closely with the \textit{Tools \& Programming} team to ensure seamless functionality 
%   within the games. Moreover, I conducted thorough \textit{code reviews} to identify areas for improvement, including 
%   performance optimizations, adherence to coding standards, and bug fixing.
% \end{itemize}

% \mediumspace

% \textit{\fontsize{12}{0}\selectfont \underline{Industry Research}}
% \begin{itemize}\setlength\itemsep{0.3em}
%   \item \textbf{Asynchronous IP-Scanning}: In this project, I designed an \textit{asynchronous} IP scanning module to 
%   pipeline the scanning of a large number (more than 100K) of IPs from internet. This module was implemented in 
%   \textit{Python} while leveraging \textit{multithreading}, \textit{socket programming}, and \textit{asyncio} techniques 
%   to maximize the performance. The automation enhanced the security posture, emphasizing the importance of continuously 
%   monitoring IP addresses, and improved the productivity of the security team by up to 50\%.

%   % The automation of the IP Scanning process significantly enhanced the security posture, 
%   % emphasizing the importance of continuously monitoring and assessing IP addresses to identify potential vulnerabilities 
%   % and threats.

%   % \item \textbf{Playbooks \& Integrations (Palo Alto Cortex XSOAR)}: I developed integration 
%   % scripts and playbooks using the \textit{Palo Alto Networks Cortex XSOAR} platform to enhance security automation and 
%   % orchestration capabilities. These contributions, made to the \href{https://github.com/xsoar-contrib}{\textit{``xsoar-contrib"}} 
%   % repository, included integrations scripts for platforms such as 
%   % \href{https://github.com/demisto/content/pull/18139}{\textit{``\textcolor{blueviolet}{DB2}"}}, 
%   % \href{https://github.com/demisto/content/pull/17845}{\textit{``\textcolor{blueviolet}{GenericSQL}"}}, and 
%   % \href{https://github.com/demisto/content/pull/17455}{\textit{``\textcolor{blueviolet}{FireMon Security}"}}. These 
%   % integrations help security teams to connect such platforms with Cortex XSOAR, enabling log collection and 
%   % reinforcing overall security.
% \end{itemize}

% \mediumspace

% \begin{itemize}\setlength\itemsep{0.3em}
%   % \item \textbf{Data Masking}: I developed a RESTful microservice to securely mask personally identifiable information 
%   % in various file formats, including CSV, Excel, mainframe, and HTML.\ I also designed an efficient masking algorithm 
%   % that pseudonymizes the users' personal data stored in flat files. This service streamlined the process of masking 
%   % thousands of files, enhancing the system performance and reducing manual intervention by 70\%.

%   % \item \textbf{Data Inquiry}: I optimized the Data Inquiry service using the \textit{asyncio framework} in Python 
%   % which enabled \textit{asynchronous execution} and effective coordination of HTTP requests. It allowed the inquiry 
%   % service to process multiple requests concurrently and reduce the turnaround time. These enhancements improved the 
%   % system efficiency by up to 50\%, allowing faster responses and better utilization of computational resources.

%   \item \textbf{Anomaly Detection}: In this cyber-defense project, I developed a novel anomaly detection agent 
%   that is capable of monitoring network traffic from multiple systems using \textit{Scapy}. The agent continuously 
%   analyzed the data in real time, identifying and flagging any suspicious or unusual activities. The agent also 
%   identified malicious IPs using \textit{K-means clustering} and a repository of malicious 
%   IPs. It triggers alerts for harmful events, providing valuable information to security analysts for further 
%   investigation and threat mitigation.

%   % \item \textbf{Incident Management System}: I developed and implemented a comprehensive incident 
%   % management system to streamline the tracking and prioritization of compromised incidents. The system 
%   % allows efficient categorization of incidents based on their severity levels, allowing the \textit{security engineering} 
%   % team to prioritize and respond promptly to critical security threats. This implementation improved the security
%   % engineering team response time by 40\% for critical incidents.
% \end{itemize}